\chapter{Alternate Code Sequences For Security}

\section{Code Sequences without PLT}

Procedure Linkage Table (PLT), see Section~\ref{plt} for detail, is
used to access external functions defined in shared object and support

\begin{description}
  \item[Lazy symbol resolution] The function address is resolved only
    when it is called the first time at run-time.
  \item[Canonical function address] The PLT entry of the external
    function is used as its address, aka function pointer.
\end{description}

The first instruction in the PLT entry is an indirect branch via the
Global Offset Table (GOT), see Section~\ref{got} for detail, entry of
the external function, which is set up in such a way that it will be
updated to the address of the function body the first time when the
function is called.  Since the GOT entry is writable, any address may
be written to it at run-time, which is a potential security risk.

\subsection{Indirect Call via the GOT Slot}

For small and medium models, different code sequence is used to avoid
PLT:

\begin{figure}[H]
\Hrule
\caption{Function Call without PLT (Small and Medium Models)}
\begin{center}
\small\code{
\begin{tabular}{|l|c|l|}
\cline{1-1}\cline{3-3}
extern void func (void);  &&.globl func\\
func (void);              &&call *func@GOTPCREL(\%rip)\\
\cline{1-1}\cline{3-3}
\end{tabular}
}
\end{center}
\Hrule
\end{figure}

The direct branch is replaced by an indirect branch via the GOT slot,
which is similar to the first instruction in the PLT slot.

\begin{figure}[H]
\Hrule
\caption{Function Address without PLT (Small and Medium Models)}
\begin{center}
\small\code{
\begin{tabular}{|l|c|l|}
\cline{1-1}\cline{3-3}
extern void func (void);  &&~~.globl func\\
void* ptr (void)          && func: \\
\{                        &&~~movq func@GOTPCREL(\%rip), \%rax  \\
~~return func;            &&~~ret \\
\}			  && \\
\cline{1-1}\cline{3-3}
\end{tabular}
}
\end{center}
\Hrule
\end{figure}

Instead using the PLT slot as function address, the function address
is retrieved from the GOT slot.

If linker determines the function is defined locally, it converts
indirect branch via the GOT slot to direct branch with a \code{nop}
prefix and converts load via the GOT slot to load immediate or
\code{lea}, see Section~\ref{opt_gotpcrelx} for details.

After dynamic linker resolved all symbols by updating GOT entries with
symbol addresses, GOT can be made read-only and overwriting GOT becomes
a hard error immediately.  Since PLT is no longer used to call external
function, lazy symbol resolution is disabled and a function can only be
interposed during symbol resolution at startup.  Tools and features which
depend on lazy symbol resolution will not work properly.  However, there
are also a few side benefits:

\begin{description}
\item[No extra direct branch to PLT entry]  Since indirect branch is 6
  byte long and direct branch is 5 byte long, when indirect branch via
  the GOT slot is used to call a local function, code size will be
  increased by one byte for each call.  Since one PLT slot has 16 bytes,
  there will be code size increase when indirect branch via the GOT slot
  is used to call an external function more than 16 times.
\item[Less memory usage] Since GOT pages are updated before program
  starts, they can be shared by multiple instances of the same ELF image.
\item[Custom calling convention]  Since external function is called
  directly via the GOT slot, instead of invoking dynamic linker to
  lookup function symbol when called the first time, parameters can be
  passed differently from what is specified in this document.
\end{description}

\subsection{Thread-Local Storage without PLT}

TLS code sequences for general and local dynamic models can be updated to
replace direct call to \code{__tls_get_addr} via the PLT entry,
with indirect call to \code{__tls_get_addr} via the GOT slot, see
Figure~\ref{__tls_get_addr}.  Since direct
\code{call} instruction is 4-byte long and indirect \code{call}
instruction is 5-byte long, the extra one byte must be handled properly.

\begin{figure}[H]
\Hrule
\caption{\code{__tls_get_addr} Call}
\label{__tls_get_addr}
\begin{center}
\myfontsize\code{
\begin{tabular}{ll|ll}
\multicolumn{2}{c}{Direct via PLT} & \multicolumn{2}{c}{Indirect via GOT} \\
\hline
call  & __tls_get_addr@PLT	& call & *__tls_get_addr@GOTPCREL(\%rip) \\
\end{tabular}
}
\end{center}
\Hrule
\end{figure}

\subsubsection{General Dynamic Model for Global Variable}

For general dynamic model, one \code{0x66} prefix before \code{call}
instruction is removed to make room for indirect call:

\begin{verbatim}
extern __thread int x;
\end{verbatim}

\noindent
the following alternate code sequence loads address of \code{x} into
\reg{rax} without PLT:

\begin{table}[H]
\Hrule
\caption{General Dynamic Model Code Sequence (LP64)}
\begin{center}
\myfontsize\code{
\begin{tabular}{lll|lll}
\multicolumn{3}{c}{With PLT} & \multicolumn{3}{c}{Without PLT} \\
\hline
0x00 & .byte & 0x66			& 0x00 & .byte & 0x66 \\
0x01 & leaq  & x@tlsgd(\%rip),\%rdi	& 0x01 & leaq & x@tlsgd(\%rip),\%rdi \\
0x08 & .word & 0x6666			& 0x08 & .byte & 0x66  \\
0x0a & rex64 &				& 0x09 & rex64 & \\
0x0b & call  & __tls_get_addr@PLT	& 0x0a & call & *__tls_get_addr@GOTPCREL(\%rip) \\
\end{tabular}
}
\end{center}
\Hrule
\end{table}

\begin{table}[H]
\Hrule
\caption{General Dynamic Model Code Sequence (ILP32)}
\begin{center}
\myfontsize\code{
\begin{tabular}{lll|lll}
\multicolumn{3}{c}{With PLT} & \multicolumn{3}{c}{Without PLT} \\
\hline
0x00 & leaq  & x@tlsgd(\%rip),\%rdi	& 0x00 & leaq & x@tlsgd(\%rip),\%rdi \\
0x07 & .word & 0x6666			& 0x07 & .byte & 0x66  \\
0x09 & rex64 &				& 0x08 & rex64 & \\
0x0a & call  & __tls_get_addr@PLT	& 0x09 & call & *__tls_get_addr@GOTPCREL(\%rip) \\
\end{tabular}
}
\end{center}
\Hrule
\end{table}

\subsubsection{Static Thread-Local Variable}

For local dynamic model, indirect call is used instead of direct call:

\begin{verbatim}
static __thread int x;
\end{verbatim}

\noindent
the following alternate code sequence loads the address of the
TLS block of the module, which contains variable \code{x}, into
\reg{rax} without PLT:

\begin{table}[H]
\Hrule
\caption{Local Dynamic Model Code Sequence (LP64)}
\begin{center}
\myfontsize\code{
\begin{tabular}{lll|lll}
\multicolumn{3}{c}{With PLT} & \multicolumn{3}{c}{Without PLT} \\
\hline
0x00 & leaq  & x@tlsld(\%rip),\%rdi	& 0x00 & leaq & x@tlsld(\%rip),\%rdi\\
0x07 & call  & __tls_get_addr@PLT	& 0x07 & call & *__tls_get_addr@GOTPCREL(\%rip) \\
\end{tabular}
}
\end{center}
\Hrule
\end{table}

\begin{table}[H]
\Hrule
\caption{Local Dynamic Model Code Sequence (ILP32)}
\begin{center}
\myfontsize\code{
\begin{tabular}{lll|lll}
\multicolumn{3}{c}{With PLT} & \multicolumn{3}{c}{Without PLT} \\
\hline
0x00 & leaq  & x@tlsld(\%rip),\%rdi	& 0x00 & leaq & x@tlsld(\%rip),\%rdi\\
0x07 & call  & __tls_get_addr@PLT	& 0x07 & call & *__tls_get_addr@GOTPCREL(\%rip) \\
\end{tabular}
}
\end{center}
\Hrule
\end{table}

\subsubsection{TLS Linker Optimization}

Since the code sequence with indirect call for general dynamic model
has the same length as the one with direct call, linker just needs to
recognize new instruction pattern to convert general dynamic access to
initial exec or local exec accesses.

\begin{description}

\item[General Dynamic to Initial Exec]
To load address of \code{x} into \reg{rax}:

\begin{table}[H]
\Hrule
\caption{GD -> IE Code Transition (LP64)}
\begin{center}
\myfontsize\code{
\begin{tabular}{lll|lll}
\multicolumn{3}{c}{GD} & \multicolumn{3}{c}{IE} \\
\hline
0x00 & .byte & 0x66 			& 0x00 & movq & \%fs:0, \%rax \\
0x01 & leaq  & x@tlsgd(\%rip),\%rdi	& 0x09 & addq & x@gottpoff(\%rip),\%rax  \\
0x09 & .byte & 0x66			&      &      & \\
0x0a & rex64 &				&      &      &  \\
0x0b & call  & *__tls_get_addr@GOTPCREL(\%rip) &      &       & \\
\end{tabular}
}
\end{center}
\Hrule
\end{table}

\begin{table}[H]
\Hrule
\caption{GD -> IE Code Transition (ILP32)}
\begin{center}
\myfontsize\code{
\begin{tabular}{lll|lll}
\multicolumn{3}{c}{GD} & \multicolumn{3}{c}{IE} \\
\hline
0x00 & leaq  & x@tlsgd(\%rip),\%rdi	& 0x00 & movl & \%fs:0, \%eax \\
0x07 & .byte & 0x66                     & 0x08 & addq & x@gottpoff(\%rip),\%rax  \\
0x08 & rex64 &				&      &      &  \\
0x0a & call  & *__tls_get_addr@GOTPCREL(\%rip) &      &       & \\
\end{tabular}
}
\end{center}
\Hrule
\end{table}

\item[General Dynamic to Local Exec]
To load address of \code{x} into \reg{rax}:

\begin{table}[H]
\Hrule
\caption{GD -> LE Code Transition (LP64)}
\begin{center}
\myfontsize\code{
\begin{tabular}{lll|lll}
\multicolumn{3}{c}{GD} & \multicolumn{3}{c}{LE} \\
\hline
0x00 & .byte & 0x66                     & 0x00 & movq  & \%fs:0, \%rax \\
0x01 & leaq  & x@tlsgd(\%rip),\%rdi	& 0x09 & leaq  & x@tpoff(\%rax),\%rax\\
0x08 & .byte & 0x66			&      &       & \\
0x09 & rex64 &				&      &       & \\
0x0a & call  & *__tls_get_addr@GOTPCREL(\%rip)	&      &       & \\
\end{tabular}
}
\end{center}
\Hrule
\end{table}

\begin{table}[H]
\Hrule
\caption{GD -> LE Code Transition (ILP32)}
\begin{center}
\myfontsize\code{
\begin{tabular}{lll|lll}
\multicolumn{3}{c}{GD} & \multicolumn{3}{c}{LE} \\
\hline
0x00 & leaq  & x@tlsgd(\%rip),\%rdi	& 0x00 & movl  & \%fs:0, \%eax \\
0x07 & .byte & 0x66			& 0x08 & leaq  & x@tpoff(\%rax),\%rax\\
0x08 & rex64 &				&      &       & \\
0x09 & call  & *__tls_get_addr@GOTPCREL(\%rip)	&      &       & \\
\end{tabular}
}
\end{center}
\Hrule
\end{table}

\end{description}

\begin{description}
\item[Local Dynamic to Local Exec]
For local dynamic model to local exec model transition, linker generates
4 \code{0x66} prefixes, instead of 3, before \code{mov} instruction
for LP64 and generate a 5-byte \code{nop}, instead of 4-byte, before
\code{mov} instruction for ILP32.  To load the address of the TLS block
of the module, which contains variable \code{x}, into \reg{rax} without
PLT:

\begin{table}[H]
\Hrule
\caption{LD -> LE Code Transition (LP64)}
\begin{center}
\myfontsize\code{
\begin{tabular}{lll|lll}
\multicolumn{3}{c}{LD} & \multicolumn{3}{c}{LE} \\
\hline
0x00 & leaq  & x@tlsld(\%rip),\%rdi	& 0x00 & .long & 0x66666666 \\
0x07 & call  & *__tls_get_addr@GOTPCREL(\%rip)	& 0x04 & movq & \%fs:0,\%rax\\
\end{tabular}
}
\end{center}
\Hrule
\end{table}

\begin{table}[H]
\Hrule
\caption{LD -> LE Code Transition (ILP32)}
\begin{center}
\myfontsize\code{
\begin{tabular}{lll|lll}
\multicolumn{3}{c}{LD} & \multicolumn{3}{c}{LE} \\
\hline
0x00 & leaq  & x@tlsld(\%rip),\%rdi	& 0x00 & nopw & 0x0(\%rax) \\
0x07 & call  & *__tls_get_addr@GOTPCREL(\%rip)	& 0x05 & movl & \%fs:0,\%eax\\
\end{tabular}
}
\end{center}
\Hrule
\end{table}

\end{description}
